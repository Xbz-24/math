\documentclass[12pt,a4paper]{article}
\usepackage{amsmath, amssymb}
\usepackage{geometry}
\usepackage{fancyhdr}
\usepackage{titlesec}
\usepackage{xcolor}
\usepackage{mathpazo} % Palatino font
\usepackage{enumitem}
\usepackage{background}

% Watermark
\backgroundsetup{
	scale=1,
	color=black!20,
	position=current page.south,
	vshift=10cm,
	contents={\Huge \textbf{University of Buenos Aires}}
}

\geometry{a4paper, margin=1in}
\pagestyle{fancy}
\fancyhf{}
\rhead{\textcolor{purple}{Math Test}}
\lhead{\textcolor{purple}{\today}}
\rfoot{Page \thepage}

\titleformat*{\section}{\large\bfseries\color{purple}}

\begin{document}
	
	\begin{center}
		\colorbox{gray!20}{
			\parbox{0.7\textwidth}{
				\centering \Huge \textcolor{purple}{Mathematics Examination}
			}
		}
	\end{center}
	\vspace{10mm}
	
	\section*{Question 1}
	Given two lines:
	\begin{align*}
		L1 &: X = \lambda(0,-1,1) + (4,2,1), \\
		L2 &: X = \lambda(0,1,2) + (4,-2,-4)
	\end{align*}
	and the point \( A = (4,2,1) \). Determine:
	\begin{enumerate}[label=\textcolor{purple}{\arabic*}.]
		\item A point \( B \) that belongs to both \( L1 \) and \( L2 \).
		\item A point \( C \) on \( L2 \) such that triangle \( ABC \) is right-angled at \( A \).
	\end{enumerate}
	
	\section*{Question 2}
	Let \( A \) be the matrix
	\[
	\begin{pmatrix}
		2 & 1 \\
		0 & -6 \\
		4 & 5 \\
	\end{pmatrix}
	\]
	and \( B \) be the matrix
	\[
	\begin{pmatrix}
		1 & -1 \\
		k & -k \\
		3 & k+1 \\
	\end{pmatrix}
	\]
	Determine all \( k \in \mathbb{R} \) for which there exist infinitely many matrices \( X \in \mathbb{R}^{2 \times 2} \) such that \( AX = BX \). For each value of \( k \), find all matrices \( X \in \mathbb{R}^{2 \times 2} \) that solve the equation.
	
	\section*{Question 3}
	Let \( \mathcal{S} = \{ x \in \mathbb{R}^4 \, | \, -x_1 + x_2 + x_4 = 0 \} \) and \( \mathcal{T} = \langle (4,1,0,2), (4,1,2,0) \rangle \). Find, if possible, a subspace \( \mathcal{W} \) of \( \mathbb{R}^4 \) that simultaneously satisfies:
	\begin{itemize}
		\item \(\dim(\mathcal{W}) = 2\),
		\item \( \mathcal{S} + \mathcal{W} = \mathbb{R}^4 \),
		\item \( \mathcal{S} \cap \mathcal{W} \subset \mathcal{T} \).
	\end{itemize}
	
\end{document}
