\documentclass{article}
\usepackage{amsmath, amssymb}

\begin{document}
	
	Let
	\[
	\alpha = 
	\left\{
	\begin{pmatrix}
		1 & 0 \\
		0 & 0
	\end{pmatrix},
	\begin{pmatrix}
		0 & 1 \\
		0 & 0
	\end{pmatrix},
	\begin{pmatrix}
		0 & 0 \\
		1 & 0
	\end{pmatrix},
	\begin{pmatrix}
		0 & 0 \\
		0 & 1
	\end{pmatrix}
	\right\}
	\]
	\[
	\beta = \{1, x, x^2\}
	\]
	\[
	\gamma = \{1\}.
	\]
	
	(a) Define \( T: M_{2\times2} (K) \to M_{2\times2} (K) \) by \( T(A) = A^t \). Calculate \([T]_{\alpha}^{\alpha}\).
	
	\textbf{Solution:}
	
	The transformation \( T \) is defined to take a matrix \( A \) and return its transpose \( A^t \). To find the matrix representation of \( T \) with respect to the basis \( \alpha \), we need to see how \( T \) acts on each matrix in \( \alpha \) and then express the result in terms of the basis \( \alpha \).
	
	Let's apply \( T \) to each matrix in \( \alpha \):
	
	1. \( T \) applied to 
	\[
	\begin{pmatrix}
		1 & 0 \\
		0 & 0
	\end{pmatrix}
	\]
	gives:
	\[
	\begin{pmatrix}
		1 & 0 \\
		0 & 0
	\end{pmatrix}
	\]
	This matrix can be expressed as a linear combination of the basis \( \alpha \) as:
	\[ \begin{pmatrix}
		1 \\
		0 \\
		0 \\
		0
	\end{pmatrix} \]
	
	2. \( T \) applied to 
	\[
	\begin{pmatrix}
		0 & 1 \\
		0 & 0
	\end{pmatrix}
	\]
	gives:
	\[
	\begin{pmatrix}
		0 & 0 \\
		1 & 0
	\end{pmatrix}
	\]
	This matrix can be expressed as:
	\[ \begin{pmatrix}
		0 \\
		0 \\
		1 \\
		0
	\end{pmatrix} \]
	
	3. \( T \) applied to 
	\[
	\begin{pmatrix}
		0 & 0 \\
		1 & 0
	\end{pmatrix}
	\]
	gives:
	\[
	\begin{pmatrix}
		0 & 1 \\
		0 & 0
	\end{pmatrix}
	\]
	This matrix can be expressed as:
	\[ \begin{pmatrix}
		0 \\
		1 \\
		0 \\
		0
	\end{pmatrix} \]
	
	4. \( T \) applied to 
	\[
	\begin{pmatrix}
		0 & 0 \\
		0 & 1
	\end{pmatrix}
	\]
	gives:
	\[
	\begin{pmatrix}
		0 & 0 \\
		0 & 1
	\end{pmatrix}
	\]
	This matrix can be expressed as:
	\[ \begin{pmatrix}
		0 \\
		0 \\
		0 \\
		1
	\end{pmatrix} \]
	
	Combining these results, the matrix representation \([T]_{\alpha}^{\alpha}\) is:
	\[ 
	\begin{pmatrix}
		1 & 0 & 0 & 0 \\
		0 & 0 & 1 & 0 \\
		0 & 1 & 0 & 0 \\
		0 & 0 & 0 & 1
	\end{pmatrix}
	\]
	
	This matrix tells us how the transformation \( T \) acts on the space of \( 2 \times 2 \) matrices with respect to the basis \( \alpha \).
	
\end{document}
